\section{Technical Details}

Efficient and relevant retrieval is crucial for user satisfaction. Below, we outline our search-algorithm design.

\subsection{Search Algorithm}

We use a fuzzy matching approach rather than exact string match. Fuzzy search handles variations and typos, which is essential for name based queries.

\subsection{Fuzzy Search}

We measure similarity via the Levenshtein distance, defined as the minimum number of single character edits (insertions, deletions, or substitutions) needed to transform one string into another:
\begin{itemize}
    \item insertion: \texttt{cot}~$\to$~\texttt{coat}
    \item deletion: \texttt{coat}~$\to$~\texttt{cot}
    \item substitution: \texttt{cot}~$\to$~\texttt{cat}
\end{itemize}

\subsection{Implementation}

We compute a weighted distance over three fields: name ($d_{\mathrm{name}}$), university ($d_{\mathrm{univ}}$), and department ($d_{\mathrm{dept}}$). The combined score is
\[
  \text{score} = 0.6\,d_{\mathrm{name}} \;+\; 0.25\,d_{\mathrm{univ}} \;+\; 0.15\,d_{\mathrm{dept}}\,.
\]
Higher weights for names reflect their shorter length and greater importance. University and department fields receive lower weights to prevent overly long strings from dominating the score.

For \emph{similar profiles}, we compute the same weighted distance between the current profile and all others, then sort by increasing score.