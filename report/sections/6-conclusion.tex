\section{Conclusion \& Future Work}

\subsection{Future Work}

The biggest limitation of our search engine is the lack of data. We only have a small subset of professors from four universities, which limits the search engine's effectiveness. To improve the search engine, we need to crawl more universities and gather more data. Not only the number of professors but also more fields such as research interests, publications, and full contact information would be beneficial. The easiest way to do this, would be to use the ORCID API, which provides information about researchers, including their publications and affiliations.

Moreover, the search engine could be enhanced by implementing more advanced search features, such as filtering by research interests, publications or their role (eg: teacher, researcher, etc.). This would also have to be implemented for the similar profiles. However, I'm not sure if fuzzy search would be enough for the more advanced search features, setting up the weights for each field and the search algorithm could require some finetuning.

Another simple but effective improvement would be to add an autocomplete feature to the search bar. Moreover, the current system stores all the data in a single json file. Creating a dedicated backend with a database would allow for more efficient data management and retrieval. This would allow us to implement profile views, top searched professors, and account management features.

Lastly, we have done minimal treatment to the data we crawled. One example is the names of the departments, which are often not standardized. For example, at INSA Lyon we have "Computer Science", at Tsinghua University we have "Department of Computer Science and Technology", and at Imperial College London we have "Department of Computing". Standardizing these names will improve search accuracy and similar profile recommendations.

\subsection{Conclusion}

In conclusion, we have built a simple search engine that allows users to find professors based on their names, universities, and departments. While functional, the system's utility depends on broader data coverage, richer profile attributes, and backend improvements. We hope this work serves as a foundation for future academic search tools.

\subsection{Ethical Use \& Privacy}
All data is sourced from publicly accessible university pages; no protected content is scraped. We exclude student records and only display faculty information. Users must refrain from misuse, such as spamming or harassment.
